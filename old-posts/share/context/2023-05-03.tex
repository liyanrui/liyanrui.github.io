\environment xxxenv
\setupfootertexts[便纵有万种风情,更与何人说?]
\defineoverlay[foo][{\externalfigure[figures/1106826218.jpg][width=.4\textwidth]}]
\defineframed[bar][frame=off,background={foo}]
\def\arrowtext#1{\text{\lower.25em\hbox{#1}}}
\starttext
\title{\bluebox{\vbox{\hbox{指数与对数函数}\hbox{\ss\small——学好数学,能让你风情万种}}}}

\startitemize[n,inmargin]
\item 为保证信息安全传输,有一种系统称为秘密密钥密码系统,其加密、解密原理如下:\\
明文 $x$ \xrightarrow[medium]{\arrowtext{加密密钥系统}} 密文 $t$ \xrightarrow[medium]{\arrowtext{发送}} 密文 $t$ \xrightarrow[medium]{\arrowtext{解密密钥系统}} 明文 $y$。现在加密密钥为 $t = 2a^{x + 1}$ $(a > 0\text{ 且 } a \ne 1)$,解密密钥为 $y = 3t - 5$,过程如下:发送方发送明文“1”,通过加密后得到密文“18”,再发送密文“18”,接收方通过解密密钥解密得明文“49”。若接收方接到明文“4”,则发送方发送的明文为\blankbar
\startoptions[m]
* -\log_3 2
* \log_3\dfrac{3}{2} + 1
* 162
* \log_3\dfrac{7}{2} - 1
\stopoptions
\vfill
\item 下列函数中,值域为 $(0, +\infty)$ 的是\blankbar
\startoptions[m]
* y = x^2
* y = \dfrac{2}{x}
* y = 2^x
* y = \|\log_2 x\|
\stopoptions
\vfill
\item 函数 $f(x) = \left(\dfrac{1}{3}\right)^{x^2 - 6x + 5}$ 的单调递减区间为\blankbar
\startoptions[m]
* \mathcal{R}
* [-3, 3]
* (-\infty, 3]
* [3, +\infty)
\stopoptions
\vfill
\item 方程 $1 + \log_2 x = \log_2 (x^2 - 3)$ 的解为\blankbar
\vfill
\stopitemize
\page
\startitemize[n, continue, inmargin]
\item 记 $S_n$ 为等差数列 $\{a_n\}$ 的前 $n$ 项和,已知 $S_4 = 0$ ,$a_5 = 5$,则\blankbar
\startoptions[m]
* a_n = 2n - 5
* a_n = 3n - 10
* S_n = 2n^2 - 8n
* S_n = \dfrac{1}{2}n^2 - 2 n
\stopoptions
\blank[4*line]
\item 关于函数 $f(x) = \sin\|x\| + \|\sin x\|$ 有下述四个结论:
\blank
\startitemize[foo,horizontal,two][stopper=]
\startitem $f(x)$ 是偶函数\stopitem
\startitem $f(x)$ 在区间 $\left(\dfrac{\pi}{2}, \pi\right)$ 单调递增\stopitem
\startitem $f(x)$ 在 $[-\pi, \pi]$ 有 4 个零点\stopitem
\startitem $f(x)$ 的最大值为 $2$\stopitem
\stopitemize
\blank
其中所有正确结论的编号是\blankbar
\startoptions
* \fooframe{1}\fooframe{2}\fooframe{4}
* \fooframe{2}\fooframe{4}
* \fooframe{1}\fooframe{4}
* \fooframe{1}\fooframe{3}
\stopoptions
\item 曲线 $y = 3(x^2 + x)e^x$ 在点 $(0, 0)$ 处的切线方程为\blankbar
\blank[4*line]
\item $\triangle ABC$ 的内角 $A$,$B$,$C$ 的对边分别为 $a$,$b$,$c$,设 $(\sin B - \sin C)^2 = \sin^2A - \sin B\sin C$。
\startitemize[n,packed][left=(, right=), stopper=]
\item 求 $A$。
\item 若 $\sqrt{2}a + b = 2c$,求 $\sin C$。
\stopitemize
\vfill
\stopitemize

\definelayer[foo][width=\paperwidth,height=\paperheight]
\setlayer
  [foo]
  [preset=leftbottom]
  {\externalfigure[figures/2023-05-03.jpg][width=.25\textwidth]}
\setupbackgrounds[page][background=foo]
\stoptext

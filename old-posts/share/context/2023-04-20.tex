\environment xxxenv
\setupfootertexts[To live would be awfully big adventure.]
\starttext
\title{\bluebox{会暖和起来的,数学!}}

\startitemize[n,inmargin]
\item 为了得到函数 $y = 2\sin\left(\dfrac{x}{3} + \dfrac{\pi}{6}\right)$,$x\in{\bf R}$ 的图像,只需把函数 $y = 2\sin x$,$x\in{\bf R}$ 的图像上所有的点\blankbar
\startitemize[A,packed]
\item 向左平移 $\dfrac{\pi}{6}$ 个单位长度,再把所得各点的横坐标缩短到原来的 $\dfrac{1}{3}$ 倍(纵坐标不变)
\item 向右平移 $\dfrac{\pi}{6}$ 个单位长度,再把所得各点的横坐标缩短到原来的 $\dfrac{1}{3}$ 倍(纵坐标不变)
\item 向左平移 $\dfrac{\pi}{6}$ 个单位长度,再把所得各点的横坐标缩短到原来的 $3$ 倍(纵坐标不变)
\item 向右平移 $\dfrac{\pi}{6}$ 个单位长度,再把所得各点的横坐标缩短到原来的 $3$ 倍(纵坐标不变)
\stopitemize
\vfill
\item 函数 $y = 4\sin\left(3x + \dfrac{\pi}{4}\right) + 3\cos\left(3x + \dfrac{\pi}{4}\right)$ 的最小正周期是\blankbar
\startoptions[m]
* 6\pi
* 2\pi
* \dfrac{2\pi}{3}
* \dfrac{\pi}{3}
\stopoptions
\vfill
\item 等差数列 $\{a_n\}$ 的公差为 $2$,若 $a_2$,$a_4$,$a_8$ 成等比数列,则 $\{a_n\}$ 的前 $n$ 项和 $S_n=$\blankbar
\startoptions[m]
* n(n+1)
* n(n-1)
* \dfrac{n(n+1)}{2}
* \dfrac{n(n-1)}{2}
\stopoptions
\vfill
\stopitemize
\page
\startitemize[n,continue,inmargin]
\item 正方体的内切球与其外接球的体积之比为\blankbar
\startoptions[m]
* 1:\sqrt{3}
* 1:3
* 1:3\sqrt{3}
* 1:9
\stopoptions
\vfill
\item 某旅游爱好者计划从 3 个亚洲国家 $A_1$,$A_2$,$A_3$ 和 3 个欧洲国家 $B_1$,$B_2$,$B_3$ 中选择 2 个国家去旅游。
\startitemize[n,packed][left=(,right=),stopper=]
\item 若从这 6 个国家中任选 2 个,求这 2 个国家都是亚洲国家的概率;
\item 若从亚洲国家和欧洲国家中各任选 1 个,求这 2 个国家包括 $A_1$ 但不包括 $B_1$ 的概率。
\stopitemize
\vfill
\hfill\hbox{\externalfigure[figures/2023-04-20.jpg][width=.2\textwidth]\kern-5em}
\stopitemize
\stoptext
